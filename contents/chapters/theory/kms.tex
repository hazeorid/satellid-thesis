\section{Knowledge Management System}
\label{sec:kms}

We first need to define what is the basic underlying system we build.
What is knowledge, knowledge management, and the micro knowledge management system.

% --------------------------------------------------
\subsection{Knowledge}

Knowledge is facts, information, and skills acquired by a person through experience or education.
It is the theoretical or practical understanding of a subject, known in a particular field or in total.
Others defined it as an awareness or familiarity gained by experience of a fact or situation.
In general, it is which we are understanding that germinates from combination of data, information, experience, and individual interpretation. \url{http://www.businessdictionary.com/definition/knowledge.html}
% \parencite{BusinessDictionary:Def:Knowledge}

In this context, it is the thing of storing and managing these kinds of knowledge in a decent friendly ecosystem that has:

\begin{easylist}
& Personalized information
& State of knowing and understanding
& Object to be stored and manipulated
& Process of applying expertise
& Condition of access to information
& Potential to influence action
\end{easylist}

% --------------------------------------------------
\subsection{Knowledge Management}

\ac{KM}\index{knowledge management} is a

% --------------------------------------------------
\subsection{Knowledge Management System}

Can be abbreviated as \ac{KMS}\index{knowledge management system}, is a method for the improvement of business process performance. It is most often used in business in applications such as information systems, business administration, computer science, public policy, and general management. Common company departments for knowledge management systems include human resources, business strategy, and information technology. \url{http://www.businessdictionary.com/definition/knowledge-management-system.html}

In this context, it is more and beyond that, because \ac{KMS} can:

\begin{easylist}
& Comprises a range of practices used in an organization to identify, create represent distribute and enable adoption to insight and experience
& Be such insights and experience comprise knowledge, either embodied in individual or embedded in organizational processes and practices.
\end{easylist}

% --------------------------------------------------
\subsection{Micro {KMS}}

It is simply defined as the simplified and modularized version of \ac{KMS}.
The micro approach allow us to start the system as small as possible but can be enlarged and adapted into any size. Unlike any other concept and implementation of commonly defined as micro knowledge management system, this procedure takes on the essence of the system in a micro way.

