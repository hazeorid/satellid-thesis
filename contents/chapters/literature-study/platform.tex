\section{Platform}
\label{sec:platform}

To enable our software operated on the server side, we need a platform based on our technology of choice.
Since we only chose JavaScript, there is Node.js, a JavaScript platform that is capable running on the server instead of just in browser.
There is also a counterpart platform of Node.js called io.js which is more quickly updated and developed by the community because of its governance model.

% --------------------------------------------------
\subsection{Node.js}

Node.js\textsuperscript{\textregistered} (sometimes only called Node) is
a platform built on Chrome's JavaScript runtime for easily building fast, scalable network applications.
Node.js uses an event-driven, non-blocking I/O model that makes it lightweight and efficient, perfect for data-intensive real-time applications that run across distributed devices.~\autocite{Joyent:2015:Node}
Node.js was originally created by Ryan Dahl.
In a simple term, Node.js is a JavaScript on the server side, but it can be run as a multiplatform within the Node.js runtime on various operating systems.
For now, Node Joyent its development progress is sponsored and regulated by Joyent Inc., a software company that specializes in application virtualization and cloud computing.
Node has its own package manager called \ac{npm}\index{node package manager} that comes bundled with Node.js.~\autocite{Vincent:2014:npm}
Node with npm can be installed by download the archive regarding the \ac{OS} or simply by using various package managers like \verb|apt-get|, \verb|pacman|, \verb|pacman|, \verb|yum|, or \verb|brew|.
By the time of this writing, Node has been released up to version \verb|0.12.0|.
Listing \autoref{lst:node-sample} shows an example of a hello world program within a simple web server in Node.js.
It can be run by executing it from the command line (assuming Node.js has already installed) like in listing \autoref{lst:node-run} then open a browser session correspond to the \ac{URL}.
Node.js can even be written in CoffeeScript like in listing \autoref{lst:node-coffee-sample}.

\begin{listing}[ht]
\caption{Node.js code sample}
\inputminted{javascript}{\dir/include/node-sample.js}
\label{lst:node-sample}
\end{listing}

\begin{listing}[ht]
\caption{Running Node.js code}
\inputminted{shell-session}{\dir/include/node-run.shell-session}
\label{lst:node-run}
\end{listing}

\begin{listing}[ht]
\caption{Node.js code sample in CoffeeScript}
\inputminted{javascript}{\dir/include/node-sample.coffee}
\label{lst:node-coffee-sample}
\end{listing}

% --------------------------------------------------
\subsection{io.js}

io.js a JavaScript platform that is a fork and originally based on Node.js, but it is also compatible with the \ac{npm} ecosystem.
The main difference is it has more predictable release cycles and using the latest update of language, API, and performance improvements.~\autocite{iojs:2015}
io.js is created because some of the communities want a difference governance (an open governance model as opposed to corporate stewardship) direction aside from current Node.js condition.
Its development speed is very much different with Node.js, io.js (initally a fork of node version \verb|0.12|) has been released up to version \verb|1.2.0| by the time of this writing.
But for the context of the development of the software, we still stick to Node.js.
