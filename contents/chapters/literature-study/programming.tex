\section{Programming Language}
\label{sec:programming}

To build a software, we need a programming language.
The programming language of choice is JavaScript with its data interchange format called JSON.

% --------------------------------------------------
\subsection{JavaScript}

\ac{JS}\index{JavaScript} is the programming language of the Web, a high-level, dynamic, untyped interpreted programming language that is well-suited to object-oriented and functional programming styles.
~\autocite{Flanagan:2011:JS}
Based on its specification, JavaScript implementation conforms to the ECMAScript\textsuperscript{\textregistered} scripting language standard.
ECMAScript itself is an object-oriented programming language for performing computations and manipulating computational objects within a host environment.~\autocite{ECMA:2011:ECMAScript}
On the Web, JavaScript is the companion programming language to \ac{HTML} the content language of \ac{DOM} and \ac{CSS} the presentation language.
Listing \autoref{lst:js-sample} shows an example of a simple hello world program in JavaScript.

\begin{listing}[ht]
\caption{JavaScript code sample}
\inputminted{javascript}{\dir/include/js-sample.js}
\label{lst:js-sample}
\end{listing}

JavaScript itself in a web page can be declared as inline, embedded, or external program with the \ac{HTML} document.
The main logic functions of it can be \ac{HTML} modificator, client-server communicator, and data storate.
JavaScript has many basic features of a programming language such as data types, variables, operators, function definition with parameters and scopes, data structures with object and array, and many others.

Another relative to JavaScript is CoffeeScript, which takes JavaScript in a simpler syntax with whitespace indentation approach rather than using brackets and semicolons.
So there's less code to write, which make it very succints, as shown in listing \autoref{lst:coffee-sample}.
CoffeeScript is a preprocessor that compiles into JavaScript, thus it doesn't replace JavaScript.
But it also has some exciting features like aliases, classes, comprehensions, chained comparisons, parallel assignments, and more.

\begin{listing}[ht]
\caption{CoffeeScript code sample}
\inputminted{javascript}{\dir/include/coffee-sample.coffee}
\label{lst:coffee-sample}
\end{listing}

% --------------------------------------------------
\subsection{JSON}

\ac{JSON}\index{JavaScript Object Notation} based on its specification, is a lightweight, text-based, language-independent data interchange format.
\ac{JSON} was inspired and derived by the object literals of JavaScript aka ECMAScript programming language, but is programming language independent.
\ac{JSON} defines a small set of structuring rules for the portable representation of structured data.~\autocite{ECMA:2013:JSON}
It is the most common alternative of text format, such as \ac{XML} or \ac{YAML}, that is both machine and human-readable but also really flexible to program.
In \ac{JSON}, there are some some forms of data types such as object, array, just value, string, and number.
Listing \autoref{lst:json-sample} shows an example of a simple object in \ac{JSON}.

\begin{listing}[ht]
\caption{JSON code sample}
\inputminted{javascript}{\dir/include/json-sample.json}
\label{lst:json-sample}
\end{listing}

JSON has been already implemented in various languages and libraries.
JSON itself has many derivatives such as \ac{EJSON}, \ac{BSON}, \ac{JSON-LD}, and other implementations.
