\section{Analysis}
\label{sec:analysis}

These analyses includes requirements, target user, specific cases of user types, and user stories.
All of them are belong into the \textit{discover} process, analyzing and writing initial requirements with acceptance criteria.

% --------------------------------------------------
\subsection{Technology Requirements}

Below is the recommended or chosen requirements for system or software that needed to minimally able to build the implementation of Satellid and then run it locally or even via the network.
These requirements for software development are not necessarily the same and may change in the future.

\begin{table}[!h]
\centering
\begin{tabular}{ |l||l| }
\hline
Category & Software  \\ \hline
\hline
%\shortstack{Data\\Format} & JSON & ENML \\ \hline
Operating System      &  Linux $3.13$ or Mac OS X $10.7$  \\ \hline
Distribution (Linux)  &  Ubuntu $14.04$ LTS  \\ \hline
Processor  &  Intel Core $2.20$ GHz   \\ \hline
RAM        &  $4$ GB                  \\ \hline
Language   &  JavaScript V8 $3.14.x$  \\ \hline
Framework  &  Meteor $1.0.x$ and Velocity $0.4.x$  \\ \hline
Platform   &  Node.js $0.10.x$            \\ \hline
Database   &  MongoDB $2.6.x$ or $3.0.x$  \\ \hline
Browser    &  Google Chrome $41$ or Mozilla Firefox $36$ \\ \hline
\end{tabular}
\caption[Chosen technology requirements]{Chosen technology requirements to build and run Satellid}
\label{table:tech-requirements}
\end{table}

% --------------------------------------------------
\subsection{Target User}

For the target user, the strategy is kind of concise: describe the desired function for particular user.
%In the business related segment, it is by using what called a \ac{BMC}\index{Business Model Canvas} to help shape the product which users would use up to customers would pay for.
Also as concerns of a wide range of market (really specify which kind of personal), Satellid implementation is actually not impartially refer to one of between the subject is a common vertical market (generally anyone, ordinary regular users and such) or diagonal market (specific types, from professionals to large industries).
Instead, there is a take of different approach.
It's found and chosen a market that called a diagonal market, which anyone can be just a user and even a paying customer.
This could cover both vertical and diagonal market within the same product.
Diagonal product can be too risky if there is no particular focus, so first a defined focus is more likely to be considered.
The general ones are individuals and variety of teams.
The specific ones are students, educators, teachers, managers, creatives, developers, designers, entrepreneurs, software and information architect, strategy consultants, research analysts, organization managers, data analysts, even \ac{CIO}, \ac{CTO}, and \ac{CEO}.
The culmination of this product is a general purpose type of tool that benefits both side of the worlds.
But as explained in the problem scope, the product can't designed for everyone in the beginning.
So only person who frequently gather and need to manage their knowledge at almost everytime is chosen.
Because who is designed for is known, ideally the analysis can be continued and later start the design.
So it back again to the general and specific ones who do that kind of particular activity in everyday life.

% --------------------------------------------------
\subsection{Specific Cases of User Types}

To be more specific who are the personal, there is a list of cases that can be possibly covered.
Because even each of them who are different to each other, they're still basically belong to personal user types.
Using Satellid approach to do knowledge management should also consider to enable features corresponds to the expected and other cases of user types that listed here:

\begin{easylist}[itemize]
  & New users need to quickly use and understand the workflow, so they can immediately store their knowledge.
  & Existing users can search for their existing stored knowledge, so they can immediately retrieve and read them.
  & Regular persons don't need to manually shout a new knowledge, so others can know what's the latest important knowledge they have.
  & Researchers or analysts need to be able collect then manage various data and research information related to their current knowledge and purpose, so they can quickly connect and filter them to be used within their experiments.
  & Engineers need to quickly construct or choose a technology or some technologies based on their prior knowledge, so they know which are the most required and important elements.
  & Developers need to be able to improve their skills in software development by knowing what are the useful logical knowledge for them, so they can be more productive and create a better software.
  & Creatives and designers need to be able to improve their skills in art or user interface and design by knowing what are the creative way to do and creative resources out there, so they can produce more refined works and good art or design.
  & Software or information architects need to quickly architect the main partial of all information available that required, so they can structure the information within working environments.
  & Entrepreneurs need to quickly see and store their knowledge about businesses and market condition, so they can decide which action to take.
  & Event speakers need to liberate and outsource the presentation and discussion resource, so they with the audience can get the collaboration to live.
  & Students need to store and manage and find knowledge easily, so they can neatly organize a preparation for study, exam, or portfolio.
  & Educators or teachers need to share their knowledge faster and effectively throughout the class or workshop, so they can focus instead on teaching and involving students to do the actual work precisely.
  & Leader or manager, in a team need to collect all of their knowledge, so they can smarten and explain the knowledge.
  & Collaborator need to know and have the prerequisites to have, so they can work together at the same pace and level.
  & Colleagues need need to quickly know what they have to know, so they can know and learn together.
  & Recruiter of a new employee need to have the recruitees' knowledge in a consise and portable format, so they can easily compare and select who's the right candidate.
  % & Boss need to know what their employees are up to in their knowledge automatically, so they don't need to ask them rigorously.
\end{easylist}

Even not mentioned explicitly again within those user types, the pattern of knowledge management will normally always be the same, to make sure those knowledge are easily managed and neatly organized across each personal.

% --------------------------------------------------
\subsection{User Stories}

Based on agile methodologies, it is possible to incorporate the defined functionality into a simple statements called user stories.
In related to the primary logic that have been created, the following user stories are the list of features along with points that must be implemented sorted by priority.
The points are estimated by conjecture the required effort, in scale $1, 2, 3, 5, 8, 13$ (from easiest/shortest to hardest/longest).
The things which need to be prioritized in particular sequential order of implementation (based on \ac{BREAD} model) are: Read, Browse, Add, Delete, Edit.

\begin{easylist}
& $1$: As a System, I need to be run on supported platform and via a network, so that I can be used by User.
& $1$: As a System, I can connect to the database, so that I can communicate with it.
& $2$: As a System, I can import the the data without the app opened.
& $1$: As a User, I can use the app via web browser.
& $2$: As a User, I can retrieve then read the knowledge that already stored.
& $3$: As a User, I can browse or search knowledge and browse the search result.
& $5$: As a User, I can add a new knowledge based on context.
& $1$: As a User, I can delete a stored knowledge.
& $3$: As a User, I can edit a stored knowledge.
\end{easylist}
