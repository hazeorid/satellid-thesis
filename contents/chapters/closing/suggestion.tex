\section{Suggestion \& Future Work}
\label{sec:suggestion-future}

It has been realized since the beginning that no system that could cover or solve a problem perfectly without any mistake.
But at least this work is going towards it, seek to do improvement.
Error and failures can always show up and should be be solved and worked out.
There are much handful of sides on Satellid that must be improved and fixed.
Indeed, there is a need to always improve both system and implementation over time, along with agile moves based on users need and advancement of technology.
That's why, Satellid is released as an open source project.
So users or other developers can customize it and patch for custom or more features.
Therefore, if community or other researcher find it interesting and helpful, partially they can help and contribute.
Either contribution for the system or content perspective.
That's also for their own benefit when the result is getting better.
The data, information, and knowledge that involved in this writing are basically open for anyone to freely use, update/modify, and share.

There could be some major issues, such as there is no custom form yet and no easy way to import and export, including backup. So there is a possible roadmap of upcoming features or abilities that Satellid need or nice to have in the future:

\begin{easylist}
& More predefined context and field.
& Can have custom form field when adding or editing
& Import and export existing knowledge, inside and outside of the system, including backup.
& Easily import from various sources and export to various formats.
& Create, edit, and delete or \ac{BREAD} a template.
& Multimedia support, upload and download image, audio, and video.
& Integration with other networks (Google+, Facebook, Twitter, etc).
& Account system with regular and admin users, including support for multiple person even up to organization.
& Mobile version.
& Change the defined context of the knowledge.
& Development with \ac{CI}.
& Image, audio, and video support.
& Transform location data into maps.
& Data visualization and reporting.
& More options and configurations to customize.
& Browser extension (Google Chrome) or plugin (Mozilla Firefox).
& Using \ac{JSON-LD} as the base data format.
& Data encryption with bcrypt.
& End to end encryption with OpenPGP.
\end{easylist}

% other things that have not covered or mentioned yet could also be the potential use of Satelllid...
