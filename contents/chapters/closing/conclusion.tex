\section{Conclusion}
\label{sec:conclusion}

Managing tons of knowledge is possible with a simple and single system of knowledge manager.
The formula is using data and template approach to classify the knowledge with its context and structure.
So if there are tons of knowledge that can be classified with context, user can easily manage them.
Naturally user can fill their knowledge collection with clue on what needed.
Also, this isn't to replace any existing system that not related to knowledge management.
If the person need to have a project management or free form notes, they need to use other than Satellid.

Implementing that kind of knowledge manager with Web technologies can be done easily now with the help of a full stack framework called Meteor.
The main building components of those technologies in Meteor are JavaScript, \ac{JSON}, Node.js, and MongoDB.

Knowledge manager can adapt to daily knowledge context and various needs because there is data schema that is very flexible enough to be modified over times.
\ac{NoSQL} database that utilize document-based (specifically \ac{JSON}) data store is one of the best schema-less data model, that simple enough to use and fit the requirement and expectation.
With it, over changing data field can be added, updated, removed corresponds on contents of the knowledge; all without the hassle of changing the initial database scheme.
After that, there is a need to give every piece of knowledge a context, naturally.
Because everytime a new knowledge is added, at the same time it's included within a context, making it structured and leverage the meaning of each of them.

Knowledge manager can naturally stucture the data into knowledge if user is provided with a prebuilt template based on their contextual needs.
Each of the templates available wrap a context or has a context frame, so naturally user can do just basic modification of their knowledge that still bounded with chosen context.

%other things that have not covered or mentioned yet could also be the potential use of Satelllid...
