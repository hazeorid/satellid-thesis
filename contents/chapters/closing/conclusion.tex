\section{Conclusion}
\label{sec:conclusion}

%[?] How to manage huge amount of knowledge we have with just a simple system?
Managing huge amount of knowledge is possible with a simple system.
The formula is using data and template approach to classify the knowledge context.
...
The most drawback in current implementation is limitation in detail, since there is.
And still, this isn't to replace any existing system that not related to knowledge management.
If the person need to have a project management or free form notes, they need to use other than Satellid.

%[?] How to design a knowledge manager that can adapt to any knowledge document type?
Knowledge manager can adapt to any knowledge document type with data schema that is very flexible enough to be modified over times.
\ac{NoSQL} database that utilize document-based (specifically \ac{JSON}) data store is one of the best schema-less data model, that simple enough to use.
With it, over changing data field can be added, updated, removed corresponds on contents of the knowledge.
After that, there is a need to give every piece of knowledge a context, the meaning of it.

%[?] How to implement a knowledge manager with web technologies?
Implementing a knoweldge manager with web technologies can be done easily now with the help of a full stack framework called Meteor.

%[?] How can knowledge manager naturally structure the data that inputted by user?
Knowledge manager can naturally stucture the data if user is provided with a prebuilt template based on their contextual needs.
