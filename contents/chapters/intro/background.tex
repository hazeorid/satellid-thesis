\section{Background}
\label{sec:background}

There are a lot of cases on why we're choosing to build a new solution in the form of a simpler knowledge manager or commonly defined as a \ac{KMS}\index{knowledge management system}.
Briefly, based by need and intend to create a new knowledge manager that allows us to have a better and more organized knowledge.

Since a few years ago, there is a vision about making a suite that make knowledge management a breeze and intuitive, not too attached to the traditional form of information management.
Especially easier and faster in condition of immersive user experience.
In condition of software architectural design, it should be modular and can be integrated to other system.
Because in the end, it should be more helpful and make the users happy, not for creator's benefit only.
Finally impactful enough to make the life of us better.

There was an attempt to store knowledge by using an software that is designed for information or note taking. System or application that are only storing and managing temporary information and designed for too much general purpose.
So frequently often you may forgot what inside it, need different tools, or can only open it properly with that specific software.
For example, when using Wikipedia, Google Docs, Evernote, we could counter those frequently often problems.
We stuck and too much dependent on it, whereas deep down essentially those are just text files.
Even though those kind of online tools can be integrated, but those data of ours are still kind of locked in a fixed circle.
Even worse, they're mosly bloated footprint in their file formats.
Despite of that, they're still the most usable tools and system because they're currently have the most needed features.
But meanwhile, Satellid works together with users in terms of collective knowledge, so we can retrieve your knowledge and connect them together with other people.

Most of the solutions offered today still have inherent styles of traditional analog notes.
That means note taking applications just digitalize the basic nature of real world notes and books, while making it faster and organized in one place.

Knowledge management itself is sometimes ironically difficult to understand and even misunderstood in most of the society.
Kind of viewed that information with note taking apps are enough to be a knowledge manager.
While actually it can be understand with the process that beyond of information taking.
Unlike information taking that too specific and few time use (only the what, when, where), knowledge management is more into understanding and collective notion that can be used everytime (the what, why, and how).
Most of the apps or systems are only storing and managing temporary information and designed for specific needs, so often you may forgot it and need different tools.
Meanwhile, Satellid work together with you in terms of knowledge, so you can retrieve your knowledge and connect them together with other people.

People know they need to have the best way to manage their knowledge or memory, they want to help enhance or extend their knowledge or memory of their need, and they should have that best tool or system then make their life better.
So there should be the simplest, easiest, fastest, and most effective way to manage knowledge whereever.
That's why existing tools should not be slow, heavy, bloated, ineffective, vendor locked-in, and anti cross-platform.
Therefore there is a need for a single system that can adapts into various kinds of purpose in terms of managing and collecting knowledge, information, or data.
A need for a full-fledged natural system rather than just digital software application.
Because managing knowledge with bookmarks, notes, todo lists, wiki, and visual dependent approach are often messy and very limited in storing various type of data.
Also too many different softwares are separated to do just their own solution for different problems, while actually have the same basic underlying principle in managing data, information, or knowledge.

If we outline the most challenging problems we met today related with data and knowledge tools, there are:

\begin{easylist}
& Most knowledge information that we know are scattered across different documents, applications, and systems.
& Most people don't know how and where to start reading information or gathering knowledge for specific task, project, or development.
& Most way of sending message like chat and email are for conversational but not for exchanging knowledge.
& Most tools are complicated or limited, also only lives in one system.
& Most text notes are not accurate, less precise, or not informative enough.
& Most systems remain just when online only and stuck within proprietary system.
& Most services are too specific, leading to more apps needed to do something that is related but outside the functionality.
& Most user experience in current available software are bad.
& Most contribution flow in Wiki (such as Wikipedia) is too ridiculously confusing and disorganized to understand for most people, make it hard to actually or have desire to contribute.
\end{easylist}

Set out from these problems, we proposed a solution.
The solution was actually formulated and born from both joy of organizing knowdlege beyond just scattered information and frustation of existing tools.
A system that craft the best experience for users in terms of use and effectiveness.
Then a system that care about computational performance in time-based and resource-based.

As mentioned earlier, the system is proposedly named Satellid (pronounced sat-el-eed). It's a play word from satellite, because satellite can send and receive information, but moreover can be knowledge. Satellid is a satellite for identified knowledge data or documents.
Satellid is like knowledge, information, or note taking apps, sites, or methods such as pen \& paper, Google Docs, Evernote, WorkFlowy, Wikipedia, Silk, just plain text, and others.
But more than that, we combine the best essentials of those worlds together in a better, simpler, and faster way.
For now, we focus on strengthen the idea of being the best knowledge organizer we need and use everytime.
But of course, we still would offer more features and benefits in the long run even after this thesis is finished.
If Satellid is compared against its most similar tools in terms of storing information and knowledge, \autoref{table:comparison} can summarize the difference.

\begin{table}[h!]
\centering
\begin{tabular}{ |c||c|c|c|c|c| }
\hline
Type         & Satellid   & Evernote      & Google Keep & Wikipedia    & Silk \\ \hline
\hline
Focus        & Knowledge  & Notes         & Notes       & Encyclopedia & Data \\ \hline
\shortstack{Data\\Format} & JSON       & ENML          & JSON        & XML/Wiki     & JSON \\ \hline
Vision       & Ubiquitous & \shortstack{Business\\Work} & Casual Use  & \shortstack{Public,\\Academic} & Visualization \\
\hline
\end{tabular}
\caption{Knowledge-related tools comparison}
\label{table:comparison}
\end{table}

The type defines characteristic of each tools: focus is the primary substance usage of it, data format is the primary system to store and exchange data inside or outside the software, and vision is both the initial and real use case.
As an important note, the initial version of Satellid may not be complete as the others currently have.
For now we still only processes text data as the main features.

Satellid was planned to be a cross platform app, but now currently only focus in the form of a web app.
To make it possible, we need the software to able to be run on both server and client.
The server side is the main logic and computation implementation as a system.
While the client side that could be a web browser also other web enabled device like smartphone, functions as the main user interface and interaction with the system that is run on the server.
How a web app works, with the connection of client and server is established, is later described in \autoref{sec:webapp}.
