\section{Problem Scope}
\label{sec:problem-scope}

The problem that we're solving is around \ac{KM}\index{knowledge management}.
The solution that we're using is to create and develop a new system to do that and fit with simple personal way of managing daily knowledge.
The vision of Satellid is going yo be used by everyone, both for personal and business needs in casual way.
For that, we're pushing the boundary beyond ordinary \ac{KM} that often only used inside organization.
Peculiarly, making it simple and easy to use and understand by most people.
Since an early product cannot be used by everyone, the product is first to be used by particular type of users.
The primary case that focused on is personal use of knowledge management, so that is the very specific coverage that will be discussed in this writing.
The prioritized personal is who regularly get and learn about any new knowledge or wanted to store their knowledge in a single system.
For the beginning, this is fine and more possible due to our current limit about resources and specifications.
The actual implementation is made as a web application that currently still focused on text format only; not images, audios, or videos.
The main features that are going to be implemented or completed first still limited around \ac{BREAD} model operation.
The main methodologies that base this development are agile, \ac{MVP}, and \ac{ATDD}.
The tools which being used are a NoSQL document database called MongoDB, JavaScript/CoffeeScript programming language, JSON data format, Node.js platform, full-stack framework called Meteor, and source code management with Git.
Finally, Satellid's basic system should be released as an open source project and under an open source license.
