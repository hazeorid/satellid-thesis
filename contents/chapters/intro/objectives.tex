\section{Objectives}
\label{sec:objectives}

Set out from these problems, there is a proposed solution.
The solution was actually formulated and born from both joy of organizing knowledge beyond just scattered data and information, also because frustation of existing tools.
A knowledge manager that craft the simple and easy experience for users in terms of use and effectiveness within structured way based on its knowledge context.

The main aim of the work is to define and develop a that new kind of knowledge manager or \ac{KMS}\index{knowledge management system} named Satellid.
It utilizes knowledge template, which use context and structure to store and managing personal daily knowledge.
The knowledge manager itself is developed and built using Web technologies.
This also classified as a knowledge storage in process of knowledge management.
Briefly, based on need and intention to create a new knowledge manager that allow person to have a better and more organized knowledge.

The system that used is geared for own personal knowledge rather than a overwhelming worldwide information.
Therefore, the content inside are actually most of the knowledge that already known by the user.
For now, the focus is on strengthen the idea of being the best knowledge organizer that needed and used everytime a knowledge or daily knowledge is received, which are tons of those can be involved.
To be more specific and focalize the initial execution and because the first tend is to make it work only for a personal needs, almost every aspect of tools and features are small with a narrow scope but can be usable and run nicely.
So to fit it right, only features that will be actually used that could be implemented first within \ac{BREAD} model operation.
More upcoming features can be considered, but for now only limited to those features.

In common cases, person who do daily personal knowledge management is helped by the records of knowledge that stored.
That stored knowledge will be particularly helpful if the system is used frequently along the time.
For more professional use, it can be a personal knowledge manager for work life related things like work assets and conditions, inside and outside organizations, internal and related networks of people, listed or historical events, catalogue of products (such as software or gadgets), and much other.
Although it seems vary, but all of them can be generalized as data in knowledge documents.
And all of those data in knowledge documents can be customized based on the contents that needed.
Knowledge documents include such data attributes like:

\begin{easylist}
& id of the data
& name or title of the person, product, company, or event
& description or tagline of them
& owner and creator of the product
& date and time as in birth, launch, or occured date
& location in descriptive location or coordinates
& links or website URL of them
& classification meta data like labels, categories, or tags
& other meta data like license and credits
& even random or extra data that not yet known
\end{easylist}

The implementation was planned to be a cross platform app, but now currently only focus in the form of a web app.
To make it possible, essentially the software is able to be run on both server like a web server and client like a web browser.
The server side is the main logic and computation implementation as a system.
While the client side that could be a web browser also other web enabled device like smartphone, functions as the main user interface and interaction with the system that is run on the server.
The paramount in matters of cooperation and liberation that Satellid using is involvement in open source movement by releasing it as an open source project under an open source license called Apache License, specifically version 2.0. Along with the development process with particular source code management called Git.
These some parts are consentaneous to the upcoming modern and great software, supporting open source movement and collaboration with other developers, users, and community around what have been build, enables a better system and software capabilities.

The vision is to make a suite that make knowledge management a breeze and intuitive, not too attached to the traditional form of information management.
Especially easier and faster in condition of intuitive user experience that simple to use.
In condition of software architectural design, it should be simple to build and deploy.
Later in the future, it should be modular and can be integrated to other system.
And since it's very open, users who usually have the ability to develop it even further can also extend the features and functionalities as much as they want.
Because in the end, it should be more helpful and make the users happy, not for creator's benefit only.
Finally impactful enough to make life of the targeted persons better.
