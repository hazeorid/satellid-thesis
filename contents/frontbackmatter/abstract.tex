\pdfbookmark[0]{Abstract}{abstract}

\begingroup
\let\clearpage\relax
\let\cleardoublepage\relax

\chapter{Abstract}
\label{chap:abstract}

\myName. \myNPM \\
\myDepTitle \\
Bachelor’s Thesis, Computer Science and Information \\ Technology Department, Faculty of Information System, Gunadarma University, 2014. \\
Keywords: \myKeywords \\
(99+ 99+ appendix)

%\vskip 5cm
%
\vfill

Knowledge in our perspective mostly always defined as the thinking foundation within our everyday life.
We use our own brain mostly for storing them, and hope as we can retrieve and remember that in our memory, when and where we need them.
Knowledge management is the de facto way to help us to manage them, because our brain mostly cannot handle them all quickly and safely when we partially forgot some parts of them.
Knowledge management can be as simple as writing the knowledge onto a paper or book, further as complex as storing the knowledge into a software or database.

Unfortunately, there is still a lot of problems around it revolving around knowledge information, especially when we forgot a memory of knowledge.
The primary purpose of this study is dedicated to discuss, provide, and explain the solution.
The need for the innovation and improvement in knowledge management and the way and method of managing knowledge rapidly and easily.

Satellid, is the proposed system and platform for the innovative way of managing any kind of knowledge.
It is defined as an open rapid micro knowledge management system and platform, by using data and template approach, utilizing JSON as the universal data or file format.
Data behaves as the content of the knowledge while template behaves as the structure of the knowledge.
Users can use and interact with the system to do create, read, update, and delete (CRUD) a data or template.
The underlying system is designed to be fast and easy using NoSQL document database CouchDB and web technology especially Node.js and JSON-LD with some frameworks and methods.
Then we managed to solve a the problem and improve on how we think and remember our life.

\vfill

Bibliography (1990-2015)

%\pagebreak

%{
%\selectlanguage{indonesian}
%\pdfbookmark[1]{Abstrak}{Abstrak}
%\chapter*{Abstrak}
%}

\selectlanguage{american}

\endgroup

\vfill

