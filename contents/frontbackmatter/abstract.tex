\pdfbookmark[1]{Abstract}{Abstract}
\begingroup
\let\clearpage\relax
\let\cleardoublepage\relax
\let\cleardoublepage\relax

\chapter*{Abstract}

!TODO!

Knowledge in our perspective mostly always defined as the thinking foundation within our everyday life.
We use our own brain mostly for storing them, and hope as we can retrieve and remember that in our memory, when and where we need them.
Knowledge management is the de facto way to help us to manage them, because our brain mostly cannot handle them all quickly and safely when we partially forgot some parts of them.
Knowledge management can be as simple as writing the knowledge onto a paper or book, further as complex as storing the knowledge into a software or database.

Unfortunately, there is still a lot of problems around it, especially when we forgot a memory of knowledge.

Most knowledge information that we know are scattered across different documents, applications, and systems.
Most people don't know how and where to start reading information or gathering knowledge for specific task, project, or development.
Most tools are complicated or limited, also only lives in one system.
Most text notes are not accurate, less precise, or not informative enough.
Most systems remain just when online only and stuck within proprietary system.
Most services are too specific, leading to more apps needed to do something that is related but outside the functionality.
Most user experience in current available software are bad.
Most contribution flow such as in Wiki is too ridiculously confusing and disorganized to understand for most people, make it hard to actually or have desire to contribute.

Those are the problems related to managing knowledge.

The primary purpose of this study is dedicated to discuss and explain the solution of those stated problems.
The need for the innovation and improvement in knowledge management, beyond ordinary and too specific tools.
The way and method of managing knowledge, rapidly and easily.

Satellid, is the proposed system and platform for the innovative way of managing any kind of knowledge with micro size rapidly.
It is basically defined as a rapid micro knowledge management system and platform, by using data and template approach.
It is scientifically defined as a modular, structured, linked, and transcluded micro knowledge management system and platform.

This thesis mainly consists of take up on things around knowledge, rapid micro knowledge management system, and Satellid.

We used a data and template approach to implement Satellid, utilizing JSON as the universal data or file format.
So as the core, just a JSON document, it has a very micro footprint.
Think of an atom that can combined or modified into molecules, organ, then organism.

Data behaves as the content of the knowledge.
Data can consists of id, title, author, date, time, location, image, audio, video, label, category, tag, URL, and else.

Template behaves as the structure of the knowledge.
Template can provisioned for use of storing categorized knowledge such as profile, CV, account credentials, social contact, list or directory (of people, places, link of websites, software, hardware, anything), guide of study and recipe, time-related such as event and calendar, documentation in form of tutorial or operational standard, financial records of assets and expenses, plain or common texts (notes, paper, presentation, story), references or bibliography (book, paper, journal), multimedia purposes, general list planning (vacation, travel, wedding, shopping).
Moreover, it can be acustomized by users, community, developers, organization, company along with their segmented rights, pricing, and license.

Users can use and interact with the system to do create, read, update, and delete (CRUD) a data or template.
Also the data or template is importable and exportable.

The underlying system is using the cutting edge, widely used, developed technology around the web especially NodeJS and JSON-LD (a JSON variant for linked data) with some frameworks, databases and methods; it is finally possible to develop and fulfilling that kind of knowledge management.
Additional in development accelerator of the system itself are including Application Programming Interface (API), functional reactive programming, JavaScript and Meteor framework, Linux with its deployment, and open source ecosystem.

The system is mainly designed to be fast and easy.
For people who want to have the best way to manage their knowledge or memory,
help enhance or extend their knowledge or memory of their need, and
have that best tool or system then make their life better.

As initial design and development of the product, it is accessible entirely on the web via modern web browsers and mobile apps via smartphones such as Android or iOS.
Moreover, it is offline mode enabled where there is no internet connection.
Practically at first, it is delivered as an open and downloadable distributed software.
so users even developers can use it directly.
Because the web enables us to distribute and communicate this system without a need for intalling apps and worrying about compatibility,
so users can use it anywhere they are.
At personal use, team, organization, school, university, company, government, and all of places where we need our knowledge.

Now is the best time to build that.
It can be done and progressively improved along the way and time.
Because activity that involves this knowledge management is happening almost everytime,
users can use it anytime they want.
From the morning, afternoon, evening, night, doing conversation, meetup, study, planning for event, future, writing for book, thesis, and all of time we use our knowledge.

We finally managed to solve a complex problem of managing knowledge especially for everyday life up to organizational issue.
This really aims to improve on how we think and remember our life.
The system would always be improved all the time and continued with community support.
In future applications, all related expansions are possible and recommended.

%\vskip 5cm
%
\vfill

Keywords: memory, database, software, knowledge management, web, nodejs, json, open source

\pagebreak

%{
%\selectlanguage{indonesian}
%\pdfbookmark[1]{Abstrak}{Abstrak}
%\chapter*{Abstrak}
%}

\selectlanguage{american}

\endgroup

\vfill



