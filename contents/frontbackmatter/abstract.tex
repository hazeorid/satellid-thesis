%\pdfbookmark[0]{Abstract}{abstract}

\begingroup
\let\clearpage\relax
\let\cleardoublepage\relax

\label{chap:abstract}
\chapter{Abstract}

\textbf{\myName.} \myNPM \\
\textbf{\myTitle} \\
\myDepTitle \\
\myThesisType, \myDepartmentLong, \myFacultyLong, \myUni, \myYear. \\
Keywords: \myKeywords \\
(\arabic{page}+ xvii+ appendix)

\hfill

\singlespacing

% Background
Knowledge in our perspective mostly always defined as the thinking foundation within our everyday life.
We use mostly our own brain for storing them, then retrieve and recall that in our memory when we need them.
Knowledge management is the de facto way to help us to manage them, because our brain mostly cannot handle them all quickly and safely when we partially forgot some parts of them.
It can be as simple as writing the knowledge onto a paper or book, further as complex as storing the knowledge into a software or database.
Unfortunately, there is still a lot of problems around it revolving around knowledge information, especially when we forgot a memory of knowledge.
% Purpose
The primary purpose of this study is dedicated to discuss, provide, and explain the problem and solution in managing knowledge.
We build a proposed system and platform for the innovative way of managing any kind of knowledge, named Satellid.
It is defined as an open knowledge manager, by using data and template approach.
Data behaves as the content of the knowledge whereas template behaves as the structure of the knowledge.
Users can use and interact with the system to do create, read, update, and delete (CRUD) a data or template.
The underlying system is designed to be fast and easy.
As the initial implementation, we made it as a web application.
The technologies behind it are NoSQL document database called MongoDB with its JSON universal data or file format, web technologies especially JavaScript programming language, and Node.js with a full-stack framework called Meteor.
Primarily for the beginning, we focus it first on solving it for a startup work environment.
% Result
We managed to solve the problem and improve on how we think and remember our life with a better way.
Colleagues can simply put and exchange their knowledge together in a single hub.
The source code itself is released as an open source project.

\onehalfspacing

\hfill

%\noindent \myTitle

\noindent Bibliography (1990-2015)

%\pagebreak
%{
%\selectlanguage{indonesian}
%\chapter*{Abstrak}
%}

\selectlanguage{american}

\endgroup

\vfill
